\documentclass[12pt]{article}
\usepackage[spanish]{babel}
\usepackage{graphicx}
\usepackage{float}
\usepackage{multicol}
\usepackage{amsmath,amssymb, amsthm}
\usepackage{apacite}
\usepackage{fancyhdr,graphicx,parskip}
\usepackage[margin=1.5 cm]{geometry}
\usepackage[utf8]{inputenc}
\usepackage[export]{adjustbox}
\usepackage{hyperref}
\usepackage{cite}


\renewcommand{\headrulewidth}{0pt}
\fancyhead[L]{
  \adjustbox{valign=c}{
    \includegraphics[width=2.5cm,raise=-1cm]{Imagenes/Universidad.png}
  }
}
\fancyhead[R]{}
\title{\textbf{\textit{\large{UNIVERSIDAD DE CÓRDOBA\\ MONTERÍA-CÓRDOBA\\ Trabajo Cálculo II\\ Programa: Ingeniería Mecánica\\Docente: Luis Javier Rubio Hernández\\}}}}
\author{Kenia Contreras Díaz}
\date{\today}

\begin{document}
hOLA BB
\maketitle
\thispagestyle{fancy}
\begin{flushleft}


\textbf{Resolver cada uno de los siguientes ejercicios}\\
\vspace{0.5cm}




\textbf{1.} Determinar si cada una de las suigientes integrales es convergente o divergente, en caso de ser posible determine el valor de la integral.
\vspace{0.5cm}


\begin{multicols}{2}
\begin{itemize}


\Large \item $\int_{0}^{+\infty }\frac{\arctan x}{2+e^{x}}dx$



\Large\item $\int_{-3}^{+\infty }\frac{1}{x^{2}+5x+6}dx$

\columnbreak


\large\item $\int_{e^{e}}^{\infty }\ln\left ( \ln x \right )dx$



\large\item $\int_{-\infty}^{+\infty }\frac{dx}{\sqrt{1+x^{4}}}dx$

\end{itemize}
\end{multicols}
\vspace{0.5cm}


\textbf{2.} Determine el área de la región limitada entre las gráficas de $y=2-x$ y $y=4-x^{2}$ y las retas $x=-2$ y $x=3$.
\vspace{0.5cm}



\textbf{3.} Determine el volumen del sólido generado. 


\begin{itemize}


    \item Al hacer girar la region en el primer cuadrante acotada por arriba por la recta $y=2$, abajo por la curva $y=2\sin{x},0\leq x\leq \frac{\pi}{2}$, y a la izquierda por el eje $y$, alrededor de la recta $y=2$


    \item al hacer girar la region en el primer cuadrante acotada por arriba por la curva $y=x^{2}$, abajo por el eje $x$, y a la derecha por la recta $x=1$, alrededor de la recta $x=-1$
    
\end{itemize}
\vspace{0.5cm}

\textbf{4.}Utilice el método de los cascarones cilíndricos para determinar el volumen de cada uno de los sólidos que se obtienen.\\




\begin{itemize}
    \item al girar alrededor del eje y las regiones acotadas por las curvas $y=2-x^{2}, y=x^{2}$ y $x=0$

    \item al girar alrededor del eje $x$ las regiones acotadas por las curvas $y=2-x, y=\sqrt{x}$ y $y=0$
    
\end{itemize}
\vspace{0.5cm}


\textbf{5.} Calcule el volumen mediante secciones transversales de cada uno de los sólidos $S$ descritos.



\begin{itemize}
    \item Un cono truncado circular recto cuya altura es $h$, base inferior de radio $R$, y radio de la parte superior $r$

    \item Una piramide truncada con base cuadrada de lado $b$, cuadrado superior de lado $a$ y altura $h$
    
\end{itemize}


\newpage


\begin{center}
    \textbf{Solución}
\end{center}
\vspace{1cm}

\textbf{Punto 1}
\vspace{0.5cm}

\textbf{1.} \large $\int_{0}^{+\infty }\frac{\arctan x}{2+e^{x}}dx$
\vspace{0.5cm}

\normalsize

Sabemos que la función de $\arctan{x}$ cuando se aproxima a $-\frac{\pi}{2}$ tiende al $-\infty$ y cuando se aproxima a $\frac{\pi}{2}$ tiende al $+\infty$, es decir:\\



$$-\frac{\pi}{2}\leq \arctan{x}\leq \frac{\pi}{2}$$

En el caso de la integral tiene sus límites de integración entre $0$ y $+\infty$, por lo tanto decimos que:\\

$$0\leq \frac{tan^{-1}}{2+e^{x}}\leq \frac{\frac{\pi}{2}}{2+e^{x}}$$

Tenemos entonces que\\


$$\frac{\pi}{2}\lim_{t \to \infty }\int_{0}^{+\infty}\frac{1}{2+e^{x}}dx$$

Procedemos a resolver la integral de forma indefinida 


$$\int \frac{1}{2+e^{x}}dx$$

\begin{align*}
    u& = 2+e^{x}\\
    du& = e^xdx\\
    u-2& = e^x\\
    (u-2)dx& = du\\
    dx&=\frac{du}{u-2}\\
\end{align*}

Realizamos la siguiente fracción parcial\\

$$\int \frac{1}{u(u-2)}du$$

$$\frac{1}{u(u-2)}=\frac{A}{u}+\frac{b}{u-2}$$

$$\left.\begin{matrix}
    1 = A(u-2)+Bu\\S
    1=Au-2A+Bu\\
1 = (A+B)u - 2A\\
A+B=0\\
-2A=1
\end{matrix}\right\}$$



Tenemos los valores para $A$ y $B$ y son $A=-\frac{1}{2}$ y $B=\frac{1}{2}$




\begin{flushleft}
    $$=-\frac{1}{2} \int \frac{1}{u}du + \frac{1}{2}\int \frac{1}{u-2}du$$
\end{flushleft}



$$=\left[-\frac{1}{2}\ln{\left(2+e^{x}\right)}+\frac{1}{2}\ln{\left(e^{x}\right)}\right]_{0}^{t}$$



$$=-\frac{\pi}{2}\lim_{t \to \infty }\left ( \left ( -\frac{1}{2}\ln \left ( 2+e^{t}\right) +\frac{1}{2}\ln \left ( t \right )\right ) -\left ( -\frac{1}{2}\ln \left ( 2+e^{0} \right )+\frac{1}{2} \ln \left ( e^{0}\right )\right ) \right )$$



$$=-\frac{\pi}{2}\lim_{t \to \infty} -\frac{1}{2}\left(\ln\left(\frac{2+e^{t}}{e^{t}}\right)\right)-\frac{1}{2}\left(\ln\left(3\right)\right)$$


$$=\frac{\pi}{4} \ln(3)$$ 


\fbox {La \Large $\int_{0}^{+\infty }\frac{\arctan x}{2+e^{x}}dx$ \normalsize es convergente y converge en $\frac{\pi}{4}\ln(3)$}



\newpage


\textbf{2.} $\int_{-3}^{\infty } \frac{1}{x^2+5x+6} dx = \int_{-3}^{-2} \frac{1}{x^2+5x+6} dx + \int_{-2}^{\infty}\frac{1}{x^2+5x+6}dx$

\vspace{0.3cm}
\begin{align*}
    (x+3)(x+2)\\
      x+3=0\\
      x=-3\\
      x+2=0\\
      x=-2
\end{align*}

\vspace{0.5cm}
=$\int_{-3}^{\infty} \frac{1}{x^2+5x+6}dx = \lim_{t \to \-2} \int_{-3}^{t}\frac{1}{x^2+5x+6}dx + \lim_{t \to \infty} \int_{-2}^{\infty}\frac{1}{x^2+5x+6}dx$


$$\lim_{t \to -2^-} \int_{-3}^{t} \frac{1}{(x+3)(x+2)}$$
\vspace{0.5cm}


Resolvemos la siguiente fracción parcial\\


$$\frac{1}{(x+3)(x+2)}= \frac{A}{x+3}+\frac{B}{x+2}$$


   $$\left.\begin{matrix}
1 = A(x+2) + B(x+3)\\ 
1=Ax+2A+Bx+3B \\ 
1=(A+B)x + 2A+3B
\end{matrix}\right\}$$

$$\left.
A+B = 0 \atop
3A + 2B= 1
\right\}$$
\vspace{0.5cm}

Tengo entonces los valores para $A$ y $B$, siendo $A=1$ y $B = -1$

$$\lim_{t \to -2^{-}} \int_{-3}^{t} \frac{1}{x+3}-\int_{-3}^{t} \frac{1}{x+2}$$

$$\left[\lim_{t \to \-2^{-}} \ln \left(x+3\right)\right]_{-3}^{t}-\left[\lim_{t \to \-2^{-}} \ln\left(x+2\right)\right]_{-3}^{t}$$
\vspace{0.5cm}


\begin{itemize}
    \item Note qué al reeplazar el segundo límite de integración queda $\ln(-3+3)$ siendo esto $0$ y $\ln(0)$ no está definido, de modo que no se puede completar la operación, y como la primera integral diverge, la segunda también lo hará es decir; 
\end{itemize}



\fbox{La \Large{$\int_{-3}^{\infty} \frac{1}{x^2+5x+6}dx$} \normalsize es divergente.}\\




\newpage



\textbf{3.} $\int_{e^{e}}^{\infty }\ln\left ( \ln x \right )dx$\\


Sabemos que la función de $ln(x)$ cuando se aproxima a $\infty$ tiende al $\infty$, es decir:\\

$$\ln{x} < \infty$$

Entonces podemos decir que:

$$\ln(\ln(x)) \leq  \ln(x)$$

Por lo que si $\int_{e^{e}}^{\infty }\ln\left ( x \right )dx$ converge o diverge entonces $\int_{e^{e}}^{\infty }\ln\left ( \ln x \right )dx$ convergerá o divergerá

Entonces: 
$$\int_{e^{e}}^{\infty }\ln\left ( x \right )dx$$

$$= \lim_{t \to \infty } \int_{e^{e}}^{t} \ln(x)dx $$

$$= \lim_{t \to \infty } \left[ x\ln(x) - x \right]^{t}_{e^e} $$

$$\lim_{t \to \infty } \left[ \left( 
t\ln(t) - t \right) - \left( e^e \ln(e^e) - e^e
 \right) \right]$$

$$= \lim_{t \to \infty } \left[ \left( 
t\ln(t) - t \right) - \left( e^e(e-1)
 \right) \right]$$

$$ = \lim_{t \to \infty } \left( 
t\ln(t)  \right) - \lim_{t \to \infty } \left( - t \right)  - \left( e^e \ln(e^e) - e^e
 \right) $$

$$ = \infty - \left( - \infty \right)  - \left( e^e \ln(e^e) - e^e
 \right) $$

$$= es~divergente$$

Como $\int_{e^{e}}^{\infty }\ln\left ( x \right )dx$ es divergente, entonces $\int_{e^{e}}^{\infty }\ln\left ( \ln x \right )dx$ también lo es.

\newpage


\textbf{4.} \Large $\int_{-\infty}^{+\infty }\frac{dx}{\sqrt{1+x^{4}}}dx$\\

\vspace{0.3cm}
\normalsize

Cuando $\frac{1}{\sqrt{1+x^{4}}}$ se aproxima a 0 cuando tienen a $-\infty$ al igual que se aproxima a 0 cuando tienen a $\infty$

$$0 \leq \frac{1}{\sqrt{1+x^{4}}} \leq 0 $$

Se sabe que:
\\
$$0 \leq \frac{1}{\sqrt{x^{4}}} \leq 0 $$

%$$\frac{1}{\sqrt{x^4}} = \frac{1}{x^2}$$

$$0 \leq \frac{1}{\sqrt{1+ x^{4}}} \leq \frac{1}{\sqrt{x^{4}}} $$

Entonces:

$$  \int_{-\infty}^{+\infty }\frac{dx}{\sqrt{x^{4}}}$$
     
$$ = \int_{-\infty}^{0}\frac{dx}{x^{2}} + \int_{0}^{+\infty }\frac{dx}{x^{2}}$$

$$= \lim_{t \to -\infty } \int_{t}^{0}\frac{dx}{x^{2}} + \lim_{t \to -\infty } \int_{0}^{+t}\frac{dx}{x^{2}}$$

$$= \lim_{t \to -\infty } \left[-\frac{1}{x} \right]_t^0 + \lim_{t \to \infty } \left[-\frac{1}{x}\right]_0^t$$

$$=\left(\lim_{x \to 0 } - \frac{1}{x}\right) - \left(\lim_{t \to -\infty} - \frac{1}{t}\right) +  \left(\lim_{t \to -\infty} - \frac{1}{t}\right) - \left(\lim_{x \to 0 } - \frac{1}{x}\right)$$

$$=(\infty) - (0) +  (0) - (\infty)$$
 



$$= es~divergente$$

si \Large $\int_{-\infty}^{+\infty }\frac{dx}{\sqrt{x^{4}}}$  \normalsize es divergente, entonces \Large $\int_{-\infty}^{+\infty }\frac{dx}{\sqrt{1+x^{4}}}dx$ \normalsize  es divergente


\newpage 

\normalsize

\textbf{Punto 2}\\

Determine el área de la región limitada entre las gráficas de $y=2-x$ y $y=4-x^{2}$ y las retas $x=-2$ y $x=3$.
\vspace{0.5cm}

Empezaremos igualando las funciones para ver donde se cortan \\

\begin{align*}
    -x^{2}+4&=-x+2\\
    -1\left(-x^{2}+4\right)&=\left(-x+2\right)\\
    x^{2}-4&=x-2\\
    x^{2}-x&-4+2\\
    x^{2}&-x-2\\
    \left(x-2\right)&\left(x+1\right)\\
    x&=2\\
    x&=-1\\
\end{align*}

Ya sabemos en qué puntos se cortan las funciones, por lo tanto procedemos a hacer una tabla de valores que inicia desde $x=-2$ hasta $x=3$

\begin{multicols}{2}

\begin{tabular}{|c|c|}
    \hline
    $x$ & $2-x$  \\
    \hline
    -2 & 4\\
    \hline
    -1 & 3\\
    \hline
    0 & 2\\
    \hline
    1 & 1\\
    \hline
    2 & 0 \\
    \hline
    3 & -1\\
    \hline
\end{tabular}


\columnbreak


\begin{tabular}{|c|c|}
    \hline
    $x$ & $-x^{2}+4$\\
    \hline
    -2 & 0\\
    \hline
    -1 & 3\\
    \hline
    0 & 4\\
    \hline
    1 & 3\\
    \hline
    2 & 0\\
    \hline
    3 & -5\\
    \hline
\end{tabular}
\end{multicols}


Quedando de la siguiente manera:\\
\vspace{0.5cm}

\newpage



Procedemos a sacar el área de la zona limitada por $x=-2$ y $x=-1$, teniendo en cuenta que  $f\left(x\right)=-x+2$ y $g\left(x\right)=4-x^{2}$.

$$\int_{-2}^{-1}-x+2-\left ( -x^{2}+4 \right )dx$$

$$=\int_{-2}^{-1}-x+2 + x^{2}-4 dx$$

$$=-\frac{1}{2}x^{2}+2x+\frac{1}{3}x^{3}-4x$$

$$=\left [-\frac{1}{2}x^{2}-2x+\frac{1}{3}x^{3} \right ]_{-1}^{-2}$$


$$=\left(-\frac{1}{2}\left ( -1\right )^{2}-2\left ( -1 \right )+\frac{1}{3}\left ( -1 \right )^{3}\right)-\left(-\frac{1}{2}\left ( -2\right )^{2}-2\left ( -2 \right )+\frac{1}{3}\left ( -2 \right )^{3}\right)$$

$$A_{1}=\frac{11}{6}$$

Seguimos con el área número 2, que sería la zona en medio del intervalo $x=-1$ hasta $x=2$, pero en este caso mi función $f\left(x\right)$ y $g\left(x\right)$ cambian, dado que ahora $f\left(x\right)=4-x^{2}$ y $g\left(x\right)=-x+2$.\\

$$\int_{-1}^{2}-x^{2}+4-\left ( -x+2 \right )dx$$

$$=\int_{-1}^{2}-x^{2}+4 + x-2dx$$

$$=\left [ -\frac{1}{3}x^{3}+4x+\frac{1}{2}x^{2}-2x \right ]_{-1}^{2}$$

$$=\left(-\frac{1}{3}\left ( 2 \right )^{3}+4\left ( 2 \right )+\frac{1}{2}\left ( 2 \right )^{2}-2\left ( 2 \right )\right)-\left(-\frac{1}{3}\left ( -1\right )^{3}+4\left ( -1 \right )+\frac{1}{2}\left ( -1 \right )^{2}-2\left ( -1 \right )\right)$$



$$A_{2}=\frac{9}{2}$$


Para la tercera área que es la zona comprendida de $x=2$ hasta $x=3$ nuevamente cambian $f\left(x\right)$ y $g\left(x\right)$ siendo  $f\left(x\right)=-x+2$ y $g\left(x\right)=4-x^{2}$.


$$=\int_{2}^{3}-x+2-\left ( -x ^{2}+4\right ) dx$$

$$=\int_{2}^{3}-x+2 +x ^{2}-4 dx$$

$$=\left[-\frac{1}{2}x^{2}+2x+\frac{1}{3}x^{3}-4x\right]_{2}^{3}$$


$$=\left(-\frac{1}{2}\left ( 3 \right )^{2}+2\left ( 3 \right )+\frac{1}{3}\left ( 3 \right )^{3}-4\left ( 3 \right )\right)-\left(-\frac{1}{2}\left ( 2\right )^{2}+2\left ( 2 \right )+\frac{1}{3}\left (2 \right )^{3}-4\left ( 2 \right )\right)$$


$$A_{3}=\frac{11}{6}$$

Así que decimos que $A_{T}=A_{1}+A_{2}+A_{3}$

$$A_{T}=\frac{11}{6}+\frac{9}{2}+{11}{6}$$

$$A_{T}=\frac{49}{6}$$
\vspace{0.5cm}



\newpage


\textbf{3.} Determine el volumen del sólido generado. 

\textbf{a.} Al hacer girar la region en el primer cuadrante acotada por arriba por la recta $y=2$, abajo por la curva $y=2\sin{x},0\leq x\leq \frac{\pi}{2}$, y a la izquierda por el eje $y$, alrededor de la recta $y=2$.


\end{figure}

$$\int_{0}^{\frac{\pi}{2}}\pi \left ( 2-2\sin x \right )^{2} dx$$

$$=\pi \int_{0}^{\frac{\pi}{2}} \left ( 2-2\sin x \right )^{2} dx$$

$$=\pi \int_{0}^{\frac{\pi}{2}} 4-2\left ( 2 \right )\left (2\sin x \right )+4\sin^{2}x dx$$

$$=\pi \left ( 4x+8 \cos x+2x-\sin \left ( 2x \right ) \right )$$

$$=\left[\pi \left ( 6x+8 \cos x-\sin \left ( 2x \right ) \right )\right]_{0}^{\frac{\pi}{2}}$$


$$=\pi \left ( 6\left ( \frac{\pi}{2} \right )+8 \cos \left ( \frac{\pi}{2} \right )-\sin \left ( 2\left ( \frac{\pi}{2} \right ) \right ) \right )- \left ( 6\left ( 0 \right )+8 \cos \left (0 \right )-\sin \left ( 2\left ( 0 \right ) \right ) \right )$$


$$=\pi \left(3\pi-8\right)$$


$$=3\pi^{2}-8\pi$$


\textbf{b.} Al hacer girar la región en el primer cuadrante acotada por arriba por la curva $y=x^{2}$, abajo por el eje $x$, a la derecha por la recta $x=1$, alrededor de la recta $x=-1$.




$$=\int_{0}^{1}2\pi \left ( 1+x \right )x^{2} dx$$

$$=2\pi \int_{0}^{1}x^{2}+x^{3} dx$$

$$=\left[2\pi\left ( \frac{1}{3}x^{3}+\frac{1}{4} x^{4}\right )\right]_{0}^{1}$$

$$=2\pi\left(\left ( \frac{1}{3}\left ( 1 \right )^{3}+\frac{1}{4} \left (1 \right )^{4}\right )-\left ( \frac{1}{3}\left ( 0\right )^{3}+\frac{1}{4} \left (0 \right )^{4}\right )\right)$$


$$=\frac{7\pi}{6}$$


\newpage

\textbf{4.}  Utilice el método de los cascarones cilíndricos para determinar el volumen de cada uno de los sólidos que se obtienen.

\textbf{a. }Al girar alrededor del eje y las regiones acotadas por las curvas $y=2-x^{2}$, $y=x^{2}$ y $x=0$


\begin{figure}[H]
    \centering
    \includegraphics[scale=0.38]{B4.jpeg}
    \caption{Sólido a revolucionar}
    \label{Solido a revolucionar}
\end{figure}


Sabiendo que $f(x)=2-x^{2};g(x)=x^{2}$


$$A(x) = f(x)-g(x)$$


$$A(x) =2-x^{2}-x^{2}$$


$$A(x) =2-2x^{2}$$


$$A=\int_{0}^{1}2\pi x(2-2x^{2})dx$$


$$A=2\pi\int_{0}^{1} (2x-2x^{3})dx$$


$$=2\pi\left ( -\frac{1}{2}x^{4}+x^{2}\right )|_{0}^{1}$$


$$=\left ( -\pi x^{4}+2\pi x^{2}\right )|_{0}^{1}$$


$$=\left ( -\pi \left ( 1 \right )^{4}+2\pi \left ( 1 \right )^{2} \right ) - \left ( -\pi \left ( 0 \right )^{4}+2\pi \left ( 0 \right )^{2} \right )$$


$$A=\pi$$


\textbf{b.} Al girar alrededor del eje $x$ las regiones acotadas por las curvas $y=2-x, y=\sqrt{x}$ y $y=0$

\begin{figure}[H]
    \centering
    \includegraphics[scale=0.38]{WhatsApp Image 2023-06-03 at 9.34.00 PM.jpeg}
    \caption{Sólido a revolucionar}
    \label{fig:enter-label}
\end{figure}

Sabiendo que $f(x)=2-x$ y $g(x)=\sqrt{x}$

\begin{align*}
f(y)&=2-y\\
g(y)&=y^{2}\\   
\end{align*}

Restamos $f(y)-g(y)$\\


$$A(y) = f(y)-g(y)$$

$$=A(y) =2-y-y^{2}$$


$$A =\int_{0}^{1} 2\pi y \left ( 2-y-y^{2} \right )dy$$

$$=2\pi \int_{0}^{1}\left ( 2y-y^2-y^3 \right )dy$$

$$=\left[ 2\pi \left ( y^2-\frac{y^3}{3}-\frac{y^4}{4} \right ) \right]_{0}^{1}$$

$$=2\pi \left ( 1-\frac{1}{3}-\frac{1}{4} \right )$$

$$=2\pi \cdot \frac{5}{12}$$

$$A=\frac{5\pi}{6}$$\\
\vspace{0.5cm}

\textbf{5.} Calcule el volumen mediante secciones transversales de cada uno de los sólidos $S$ descritos.\\

\vspace{0.5cm}

\textbf{a.} Un cono truncado circular recto cuya altura es $h$, base inferior de radio $R$, y radio de la parte superior $r$.\\
\vspace{0.5cm}


Tenemos que hallar la ecuación de la recta que pasa por los puntos $(R,o)$ y de $(r,h)$ así que para ello hallaremos la pendiente.

$$m=\frac{0-h}{R-r}=-\frac{h}{R-r}$$

Tenemos que la ecuación de la recta es:\\

$$y-o=m\left(x-R\right)$$

$$y=-\frac{h}{R-r}\left(x-R\right)$$

$$y=\frac{h}{R-r}\left(R-x\right)$$

$$x=R-\frac{\left(R-r\right)}{h}y$$

Reemplazamos en la siguiente fórmula:\\
\vspace{0.5cm}


$$\int_{0}^{h}\pi \left ( f(x) \right )^{2}dx$$

$$=\int_{0}^{h}\pi \left[R-\frac{(R-r}{h}y\right]^{2}$$

$$=\pi\int_{0}^{h}\left[r^{2}-\frac{2R(R-r)}{h}y+\left(\frac{R-r}{h}\right)^{2}y^{2}\right]dy$$

$$=\pi\left[r^{2}y-\frac{r(R-r)}{h}y^{2}+\frac{1}{3}\left(\frac{R-r}{h}\right)^{2}y^{3}\right]_{0}^{h}$$

$$=\pi\left[hR^{2}-R(R-r)h+\frac{/R-r)^{2}}{3}h\right]$$

$$=\pi h\left(R^{2}-R(R-r)+\frac{1}{3}(R-r)^{2}\right)$$


$$=\pi h\left(R^{2}-R^{2}+Rr+\frac{1}{3}R^{2}-\frac{2}{3}Rr+\frac{1}{3}r^{2}\right)$$

$$=\pi h \left(\frac{1}{3}R^{2}+\frac{1}{3}Rr+\frac{1}{3}r^{2}\right)$$

$$=\frac{1}{3}\pi h\left(r^{2}+Rr+r^{2}\right)$$


\textbf{b.}Una piramide truncada con base cuadrada de lado $b$, cuadrado superior de lado $a$ y altura $h$

Hallaremos la ecuación que pasa por los puntos $(\frac{a}{2}.h)$ y $(\frac{b}{2},0)$.

$$m=\frac{h-0}{\frac{a-b}{2}}=\frac{2h}{a-b}$$

$$y-0=\frac{2h}{a-b}\left(x-\frac{b}{2}\right)$$

$$x=y\frac{a-b}{2h}+\frac{b}{2}$$

$$=\int_{0}^{h}\left[2\left(\frac{y(a-b)}{2h}+\frac{b}{a}\right)\right]dy$$


$$=\int_{0}^{h}\left(\frac{y(a-b)}{h}+b\right)^{2}dy=$$


$$=\int_{0}^{h}\frac{y^{2}(a-b)^{2}}{h^{2}}+\frac{2b(a-b)y}{h}+b^{2}~dy$$


$$=\left [ \frac{(a-b)^{2}y^{3}}{3h^{2}} +\frac{b(a-b)y^{2}}{h}+b^{2}y\right ]_{0}^{h}$$


$$\frac{1}{3}(a-b)^{2}h+b(a-b)h+b^{2}h$$


$$\frac{1}{3}(a-b)^{2}h+h(ab-b^{2}+b^{2})$$


$$\frac{1}{3}(a-b)^{2}h+\frac{3}{3}h(ab)$$

$$\frac{1}{3}h(a^{2}+ab+b^{2})$$


Ahora, al saber las áreas, tanto superior como inferior $A_{1}=a^{2}~A_{2}=b^{2}$ decimos que 


$$\sqrt{a^{2}b^{2}}=\sqrt{A_{1}A_{2}}$$


Así que tenemos que el volumen es: 


$$V=\frac{1}{3}h(A_{1}+A_{2}+\sqrt{A_{1}A_{2}})$$







\end{flushleft}
\end{document}
